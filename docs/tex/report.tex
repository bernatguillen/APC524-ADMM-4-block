\documentclass[paper=a4, fontsize=11pt]{scrartcl}
\usepackage[T1]{fontenc}
\usepackage{fourier}

\usepackage[english]{babel}															% English language/hyphenation
\usepackage[protrusion=true,expansion=true]{microtype}	
\usepackage{amsmath,amsfonts,amsthm} % Math packages
\usepackage[]{algorithm2e}
\usepackage[pdftex]{graphicx}	
\usepackage{url}


%%% Custom sectioning
\usepackage{sectsty}
\allsectionsfont{\centering \normalfont\scshape}


%%% Custom headers/footers (fancyhdr package)
\usepackage{fancyhdr}
\pagestyle{fancyplain}
\fancyhead{}											% No page header
\fancyfoot[L]{}											% Empty 
\fancyfoot[C]{}											% Empty
\fancyfoot[R]{\thepage}									% Pagenumbering
\renewcommand{\headrulewidth}{0pt}			% Remove header underlines
\renewcommand{\footrulewidth}{0pt}				% Remove footer underlines
\setlength{\headheight}{13.6pt}


%%% Equation and float numbering
\numberwithin{equation}{section}		% Equationnumbering: section.eq#
\numberwithin{figure}{section}			% Figurenumbering: section.fig#
\numberwithin{table}{section}				% Tablenumbering: section.tab#


%%% Maketitle metadata
\newcommand{\horrule}[1]{\rule{\linewidth}{#1}} 	% Horizontal rule

\title{
		%\vspace{-1in} 	
		\usefont{OT1}{bch}{b}{n}
		\normalfont \normalsize \textsc{APC524 Project Report} \\ [25pt]
		\horrule{0.5pt} \\[0.4cm]
		\huge ADMM 4-block solver implementation on Python \\
		\horrule{2pt} \\[0.5cm]
}
\author{
		\normalfont 								\normalsize
        Bernat Guillen, Michael Tarczon, Yuan Liu\\[-3pt]		\normalsize
        \today
}
\date{}


%%% Begin document
\begin{document}
\maketitle
\section{Introduction}
This is the project report for the final project of APC524. Our group worked on the implementation of a 3-block ADMM solver based on [cit required]. This solver is designed for problems of the form:
\begin{equation}
\label{eqCP}
	\text{max}\{\langle -c,x\rangle | \mathcal{A}_E x = b_E\;,\;\mathcal{A}_I x\geq b_I, \;x\in \mathcal{K},\; x\in\mathcal{K}_p\}
\end{equation}

Note: The inequality constraints can be removed adding slack variables.

This kind of problem generalizes SDP, DNNSDP, LP, SOCP among others. In the scope of this project only special solvers for SDP and DNNSDP have been done, the rest has to be input by hand. 

In (\ref{eqCP}), $\mathcal{K}$ is any kind of convex cone, i.e. any subset of a vector field $\mathcal{X}$ that is closed under linear combinations with positive coefficients. $\mathcal{K}_p$ is any polyhedral cone generated by the vectors $\{v_1,\dots,v_l\}$, i.e. $\mathcal{K}_p = \{k_1v_1 + \dots + k_l v_l | k_i \geq 0\}$.

The dual of the problem (\ref{eqCP}) is:

\begin{equation}
\label{eqCPdual}
\text{min}\{-\langle b_I , y_I\rangle -\langle b_E , y_E\rangle | s + \mathcal{A}^{*}_{I} y_I + z - \mathcal{A}^{*}_{E} y_E = c, s\in \mathcal{K}^*, z \in \mathcal{K}_p^*, y_I \geq 0\}
\end{equation}

In (\ref{eqCPdual}), $\mathcal{K}^*$ stands for the dual of a cone, i.e. $\{d \in \mathcal{X} | \langle d,x \rangle \geq 0 \forall x \in mathcal{K}$, and $\mathcal{A}_i^*$ denotes the adjoint of $\mathcal{A}$ (usually the transpose unless the base of the space is not orthonormal). 

Skipping all the intermediate steps (that can be found in [ref]), we write the augmented Lagrangian of the problem (after including slack variables), which is key to develop the method:

\begin{align*}
L_{\sigma} (s,z,y_E;x) := & \delta_{\mathcal{K}^*}(s) + \delta_{\mathcal{K}_p^*}(z) + \langle -b_E,y_E \rangle  + \langle x,s+z+\mathcal{A}_E^* y_E - c\rangle \\
& + \frac{\sigma}{2}||s+z+\mathcal{A}_E^* y_E - c||^2 
\end{align*}

Most of the current solvers for conic programming problems use the well-known Interior Point Methods or (as this project) Alternating Direction Method of Multipliers (ADMM). However, neither the Interior Point Methods nor the ADMM are proven to converge to the optimal solution when there are three blocks of constraints (that is, the two cones and the equality constraints). In this project we implement a variation of ADMM that has a proof of convergence.

Roughly speaking, ADMM consists on optimizing the variables in an alternating way, that is to say (in the case of a 2-block ADMM):
\begin{enumerate}
\item Update dual variable 1 assuming the rest are all fix
\item Update dual variable 2 assuming the rest are all fix
\item Update primal linear variable assuming the rest are all fix
\end{enumerate}
 
In this case, we can be more specific with the algorithm, and as seen in [ref], it proceeds as follows:
\begin{algorithm}
\KwData{$\sigma > 0$, $\tau >0$, $s^0 \in \mathcal{K}^*$, $z^0 \in \mathcal{K}_p^*$, $x^0$ such that $\mathcal{A}_Ex^0 = b_E$. $y^0_E = (\mathcal{A}_E\mathcal{A}_E^*)^{-1}\mathcal{A}_E(c-s^0-z^0)$.}
\While{k < nsteps and err > tol}{
	$s^{k+1} = \text{arg min} L_\sigma(s,z^k,y_E^k;x^k)=\Pi_{\mathcal{K}^*}(c-z^k-\mathcal{A}_E^*y_E^k-\sigma^{-1}x^k)$\;
	$y^{k+1/2}_E = \text{arg min} L_\sigma(s^{k+1},z^k,y_E;x^k) = (\mathcal{A}_E\mathcal{A}_E^*)^{-1}\mathcal{A}_E(c-s^{k+1}-z^k)$\;
	$z^{k+1} = \text{arg min} L_\sigma(s,z^k,y_E^k;x^k)=\Pi_{\mathcal{K}^*_p}(c-s^{k+1}-\mathcal{A}_E^*y_E^{k+1/2}-\sigma^{-1}x^k)$\;
	$y^{k+1}_E = \text{arg min} L_\sigma(s^{k+1},z^{k+1},y_E;x^k) = (\mathcal{A}_E\mathcal{A}_E^*)^{-1}\mathcal{A}_E(c-s^{k+1}-z^{k+1})$\;
	$x^{k+1} = x^k + \tau\sigma(s^{k+1} + z^{k+1} + \mathcal{A}_E^*y_E^{k+1} - c)$;
	}
	\caption{Algorithm Conic-ADMM3c}
\end{algorithm}

Where $\Pi_\mathcal{C}(x)$ is the projection onto $\mathcal{C}$ of $x$. In [ref] and in our code we make extensive use of Moreau's decomposition theorem that states $x = \Pi_{\mathcal{C}}(x)+ \Pi_{\mathcal{C}^*}(-x)$ or, in other words, $\Pi_{\mathcal{C}^*}(x) = x + \Pi_{\mathcal{C}}(-x)$.


\subsection{Heading on level 2 (subsection)}
Lorem ipsum dolor sit amet, consectetuer adipiscing elit. 
\begin{align}
	A = 
	\begin{bmatrix}
	A_{11} & A_{21} \\
  	A_{21} & A_{22}
	\end{bmatrix}
\end{align}
Aenean commodo ligula eget dolor. Aenean massa. Cum sociis natoque penatibus et magnis dis parturient montes, nascetur ridiculus mus. Donec quam felis, ultricies nec, pellentesque eu, pretium quis, sem.

\subsubsection{Heading on level 3 (subsubsection)}
Nulla consequat massa quis enim. Donec pede justo, fringilla vel, aliquet nec, vulputate eget, arcu. In enim justo, rhoncus ut, imperdiet a, venenatis vitae, justo. Nullam dictum felis eu pede mollis pretium. Integer tincidunt. Cras dapibus. Vivamus elementum semper nisi. Aenean vulputate eleifend tellus. Aenean leo ligula, porttitor eu, consequat vitae, eleifend ac, enim.

\paragraph{Heading on level 4 (paragraph)}
Lorem ipsum dolor sit amet, consectetuer adipiscing elit. Aenean commodo ligula eget dolor. Aenean massa. Cum sociis natoque penatibus et magnis dis parturient montes, nascetur ridiculus mus. Donec quam felis, ultricies nec, pellentesque eu, pretium quis, sem. Nulla consequat massa quis enim. 


\section{Lists}

\subsection{Example for list (3*itemize)}
\begin{itemize}
	\item First item in a list 
		\begin{itemize}
		\item First item in a list 
			\begin{itemize}
			\item First item in a list 
			\item Second item in a list 
			\end{itemize}
		\item Second item in a list 
		\end{itemize}
	\item Second item in a list 
\end{itemize}

\subsection{Example for list (enumerate)}
\begin{enumerate}
	\item First item in a list 
	\item Second item in a list 
	\item Third item in a list
\end{enumerate}
%%% End document
\end{document}